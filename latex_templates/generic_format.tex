\documentclass[12pt,a4paper]{article}
\usepackage[utf8]{inputenc}
\usepackage{amsmath}
\usepackage{amsfonts}
\usepackage{gensymb}
\usepackage{amssymb}
\usepackage{graphicx}
\usepackage{caption}
\usepackage{booktabs}
\usepackage{subfig}
\usepackage{float}
\usepackage{mhchem}
\usepackage[left=3.5cm,right=2cm,top=2cm,bottom=2.5cm]{geometry}
\usepackage[backend=biber, style=numeric, sorting=none]{biblatex}
\addbibresource{refs.bib}
\begin{document}

\section{Objectives}
The purpose of this experiment was to ---
\begin{enumerate}
        \item Learn about the microstructure of TMT steel.
		\item Learn about the basics of cold mounting process.        
        \item Learn about some precautions involved in the preparation of the sample.
\end{enumerate}

\section{Introduction}

Dead-soft steel and High-carbon steel are situated at the ends of the plain carbon steel spectrum. Dead-soft steel has 0.03\% - 0.15\% Carbon content. Wheras, High carbon steels have 0.65\% - 1.7\% or 2\% Carbon content. Because of their difference in carbon content their microstructure is also affected. The dead-soft steel is also called hypo-eutectoid steel and the remainder being Hyper-eutectoid steel up until 2.17\% Carbon content. The explanation of the microstructure can be derived from $Fe$-$Fe_{3}C$ phase diagram.\cite{cal}

\begin{figure}[H]
	\centering
	\includegraphics[height=4.2in]{fe-fe3c.png}
	\caption{The $Fe$-$Fe_{3}C$ phase diagram.\cite{cal}}
	\label{pd}
\end{figure}

From the iron-iron carbide phase diagram shown in figure \ref{pd}, it can be seen that at 727\degree C and 0.76\%C composition gives an eutectoid reaction \ce{$\gamma$ (austenite) <=> $\alpha$ (ferrite) + Fe3C (cementite)}. A type of steel called Pearlite forms at that point which has a lamellar structure with repeating layers of ferrite and cementite. For Hypoeutectoid steel the composition is below the eutectoid composition (Dead-soft steel falls into this category). When solidification occurs for a hypo-eutectoid alloy the pearlite phase starts forming in between the grain boundaries, the neucleus grows as the temperature approches the eutectoid temperature. When the alloy reaches the eutectoid temerature the transformation of $\gamma$ (austenite) into pearlite begins. After the reaction all the austenite phases become pearlite with channels of $\alpha$ (ferrite) in between (along the grain boundaries). However, since Dead-soft steel has a very low amount of carbon the microstructure will be almost all ferrite phase with very little amount of pearlite phase.

The Hypereutectoid steel shows the same phenomena but instead of ferrite forming along the grain boundaries \ce{Fe3C} forms.\cite{camp}

\begin{figure}[H]%
    \centering
    \subfloat[Equilibrium cooling of Hypo-eutectoid steel.]{{\includegraphics[width=9cm]{hpo_st.png} }}%
    \quad
    \subfloat[Equilibrium cooling of Hyper-eutectoid steel.]{{\includegraphics[width=8cm]{hpr_st.png} }}%
    \caption{Formation of Hypo-eutectoid and Hyper-eutectoid steel's microstructure.}%
    \label{hhp}%
\end{figure}

\section{Experimental Methodology}
\begin{enumerate}
\item The already sectioned sample provided by the lab, was ground using different grades of emery paper. When moving to a finer grade paper the sample's surface was cleaned and the direction of grinding was made perpendicular. The finest paper was used last.
\item When a smooth, flat surface with scratches running in the same direction was observed, the grinding was stopped. The sample was then rinsed with water and dried to prepare for polishing.
\item The polishing was done perpendicular to the direction of the last grind. The process was stopped when a smooth, flat and mirror-like surface was produced.
\item The sample was then etched for 5-6 seconds using a 2\% Nital solution (\ce{2\% HNO3 + 98\% Ethanol}). Then it was immediately washed with water and dried after using acetone on the surface.
\item Then the sample was taken under a microscope to examine.
\item After observation, schematic diagrams of both the structures were drawn.
\begin{figure}[H]
\centering
\includegraphics[scale=1]{Untitled.png} 
\caption{Schematic drawings of Dead-soft (left) and High-carbon steel's microstructure.}
\end{figure}
\end{enumerate}

\section{Discussion and Conclusion}
\subsection{Precautions}
\begin{enumerate}
\item The grinding was done carefully with back and forth motion, so that all the scratches were running in the same direction. To avoid the creation of two or more distinct surfaces on the sample, pressure was applied evenly. 
\item When moving to a finer grade emery paper it was made sure that the abrasive particles from the previous paper were completely removed. It was done by cleaning the sample's surface. The direction of grinding was also made perpendicular with each change.  
\item Before polishing, the sample was cleaned thoroughly with water. The polishing was done with minimal pressure applied on the sample. The sample was held firmly to avoid accidents.
\item Etching was done carefully as to not over-etch or under-etch the sample. If the sample was over or under etched the polishing process was done again as well as etching.
\item If the sample was shaped unevenly then gum was used for creating a suitable slope or angle for microscopy.   
\end{enumerate}
%\subsection{Grain Size Calculation}
%\begin{figure}[H]
%	\centering
%	\includegraphics[height=3.5in]{Gp_A6.jpg} 
%	\caption{The microstructure of the sample of group A-06.(Low-carbon steel)}
%	\label{ms0}
%\end{figure}
%Since figure \ref{ms0} does not show grain boundaries clearly. Group B05's sample's microstructure will be used for the calculations.
%\begin{figure}[H]
%	\centering
%	\includegraphics[height=3.7in]{Gp_B5.jpg} 
%	\caption{The microstructure of the sample of group B-05.(Low-carbon steel)}
%	\label{ms1}
%\end{figure}
%
%\begin{figure}[H]
%	\centering
%	\includegraphics[height=3.7in]{Gp_B2.jpg} 
%	\caption{The microstructure of the sample of group B-02.(Medium-carbon steel)}
%	\label{ms2}
%\end{figure}

%It is known that,
%\begin{equation}
%\textit{Grain Size,}\,D\,=\,\frac{\textit{Length of the line}}{\textit{Number of grains cut by the line}}
%\label{e1}
%\end{equation} 
%
%The length of the lines are 4235$\mu$m.
%
%From figure \ref{ms1} it can be seen that; the lines cut 29, 30 and 27 grains respectively. Using equation \ref{e1}: 
%$$D=\frac{4235}{\frac{29+30+27}{3}}=147.733\mu\textit{m}$$\\
%Similarly, from figure \ref{ms2} it can be seen that,
%$$D=\frac{4235}{\frac{51+47+46}{3}}=88.229\mu\textit{m}$$\\ 

%\subsection{Estimated Compostiton}
%From figure \ref{ms1} (Low-carbon Steel) it can be estimated that 90\% of the phases are Ferrite and the remaining 10\% is Pearlite.
%Ferrite has 0.008\% Carbon content whereas Pearlite has 0.87\%
%Carbon content. So, an estimate about the Carbon content of the whole sample can be made.
%$$\textit{Total Carbon Content}\,=\,(0.008\times0.90)+(0.87\times0.10)=0.0942\,\%$$
%This falls into the criteria for Low-carbon steel.\\
%
%From figure \ref{ms2} (Medium-carbon Steel) the estimation is almost 50\% Ferrite and 50\% Pearlite. So,
%$$\textit{Total Carbon Content}\,=\,(0.008\times0.50)+(0.87\times0.50)=0.439\,\%$$
%Which agrees with the carbon content needed for a Medium-carbon steel.

%\subsection{Strength Calculation}
%\subsubsection{From Hall-Petch Equation}
%\begin{figure}[H]
%\centering
%\includegraphics[height=3in]{hall_p_fe.png} 
%\caption{Hall-Petch relation for low carbon steel.}
%\label{hall}
%\end{figure}
%The Hall-Petch equation:
%\begin{equation}
% \sigma_{y}=\sigma_{o}+k_{y}\times d^{-1/2}
% \label{e2}
%\end{equation} 

%For Low-carbon Steel \cite{will}; $k_{y}=0.74 \frac{\textit{MPa}}{m^{-1/2}}$ and $\sigma_{o} = 70\textit{MPa}$\\
%Using equation \ref{e2}: $$\textit{Yield Strength, }\sigma_{y}=70+0.74\times (\frac{147.733}{10^{6}})^{-1/2} \approx 130.88 \textit{MPa}$$\\
%
%
%For Medium-carbon Steel; $k_{y}=0.74 \frac{\textit{MPa}}{m^{-1/2}}$ and $\sigma_{o} = 70\textit{MPa}$\\
%Using equation \ref{e2}: $$\textit{Yield Strength, }\sigma_{y}=70+0.74\times (\frac{88.229}{10^{6}})^{-1/2} \approx 148.78 \textit{MPa}$$
%
%%\begin{figure}[H]
%%\centering
%%\includegraphics[height=4in]{all_c.png} 
%%\caption{Change of the yield strength of steel for increasing carbon content.}
%%\label{all_c}
%%\end{figure}
%%
%%For Medium-carbon steel; The linear portion of the curve shown in figure \ref{all_c} can be used. It can be deduced that the slope of the straight line is $\frac{25-23}{0.003-0.002} = 2000\frac{ksi}{\%Carbon} = 13789.5181 \frac{MPa}{\%Carbon}$. So, a 0.049\% Carbon Steel will have a yield strength of  
%
%%\subsubsection{From Estimated composition}
%It is known that Ferrite has a strength of 40000 psi or 275.79 MPa and Pearlite has a strength of 120000 psi or 827.37 MPa.
%
%This information can be utilized to find out the strength of the samples.\\
%
%\textbf{Low-carbon Steel}:
%$$\sigma_{y}=(0.90\times275.79)+(0.10\times827.37)\approx330.948\textit{MPa}$$
%
%\textbf{Medium-carbon Steel}:
%$$\sigma_{y}=(0.50\times275.79)+(0.50\times827.37)\approx551.58\textit{MPa}$$
%%The difference between the result is likely because of the values of $\sigma_{o}$ and $k_{y}$ is different for every metal. Even a tiny change in carbon content drastically changes the yield strength of steel.
%
%This calculation signifies that an increase in steel's carbon content also increases its yield strength.
%
\subsection{Microstructural Features}
\begin{figure}[H]%
    \centering
    \subfloat[Microstructure of the sample given to Group-A1.]{{\includegraphics[width=8.5cm]{GP_A1.jpg} }}%
    \quad
    \\
    \subfloat[Microstructure of the sample given to Group-A6.]{{\includegraphics[width=8.5cm]{GP_A6.jpg} }}%
    \caption{Microstructures observed in experiment 3.}%
    \label{mic}%
\end{figure}
From figure \ref{mic}, it is evident that image (a) shows the microstructure of a high carbon steel sample and the other shows a dead-soft steel sample. Here, the dead soft steel sample shows a tiny amount of pearlite phase (the rest is ferrite) whereas the high carbon steel sample has almost all pearlite phase with \ce{Fe3C} phase along the grain boundaries.
\subsection{Differences between Different Types of Plain Carbon Steel} 

There are four categories of plain carbon steels :
1. Dead Soft Steel,
 2. Mild or Low Carbon Steel,
 3. Medium Carbon,
 and 4. High Carbon Steel.\\
 
They can be easily distinguished by the pearlite phases that they contain. Their microstructures can also be estimated using the Iron-Carbon phase diagram shown in figure \ref{pd}. If vertical lines are drawn at compositions 0.1\% C (Dead-soft), 0.2\% C (Mild-steel), 0.4\% C (Medium-carbon) and 1.5\% C (High-carbon); The lever rule can be applied at different temperatures to determine the relative amount of phases. 800\degree C and 600\degree C Temperatures can be used as examples because they are respectively above and below the eutectoid temperature where the \ce{$\gamma$ <=> $\alpha$ + Fe3C} reaction takes place. The phases formed above the eutectoid temperature stays intact during the transformation and the rest gets transformed into pearlite with repeating layers of ferrite and cementite. After etching this phase becomes darkened and can be used to determine the grade of the steel.\cite{camp}


\begin{figure}[H]%
    \centering
    \subfloat[High-carbon Steel. (observed by group A-1 in experiment 3)]{{\includegraphics[width=7cm]{GP_A1.jpg} }}%
    \quad
    \subfloat[Dead-soft Steel. (observed by group A-6 in experiment 3)]{{\includegraphics[width=7cm]{GP_A6.jpg} }}%
    \\
    \subfloat[Medium-carbon Steel. (observed by group B-2 in experiment 2)]{{\includegraphics[width=7cm]{Gp_B2.jpg} }}%
    \quad
    \subfloat[Low-carbon Steel. (observed by group B-5 in experiment 2)]{{\includegraphics[width=7cm]{Gp_B5.jpg} }}%
    \caption{Microstructures of different types of Plain-carbon Steel.}%
    \label{mics}%
\end{figure}

The images above show compliance with the theoretical amount of pearlite derived from the Iron-Carbon phase diagram.
\subsection{Conclusion}

In conclusion, the experiments conducted on the different types of steel samples were important for learning about the microstructures. The experiments also taught various techniques and precautions used for preparation of metallic sample. They also helped to gain insights about the correlation of various properties of materials with their microstructure.
\cleardoublepage
\printbibliography
\end{document}
